
\section*{全曲の概要}
\addcontentsline{toc}{section}{全曲の概要}

本歌曲集は1877年3月にWienで作曲された\cite{library}\cite{denki}. 出版は同年7月から8月にかけて, 作品69から作品72までの歌曲シリーズとしてSimrock社から\cite{cd}. 
交響曲第1番の初演が1876年11月で, BrahmsがSimrockへ交響曲の楽譜を送付したのが1877年5月であるから,
交響曲第1番の改訂の最中に作曲されたことになる.
なお, 後述するように第1曲「春は優しい恋の季節」の一節が交響曲第2番 (1877年夏に作曲された) に顔を出す.

前後の歌曲集 (作品70, 作品72) が数年に渡って作曲された歌曲の集まりである (1877年に作曲されたものが大部分だが) こと\cite{taiyaku}と比較すると,
この歌曲集はごく短期間にまとまって作られたものであると言える.
内容的にもこの5曲はすべて恋愛事を題材としている.

この歌曲集のなかで最も有名な第5曲「愛の歌」はLudwig Hölty (1748--1776) の詩によっており,
同じ詩に対してSchubertやMendelssohnも付曲している.
また, Karl Candidus (1817--1872) による第3曲「秘め事」も絶妙な音楽的効果のために評価が高い.
なお, Clara Schumannが本歌曲集に関してBrahmsに感想を述べた手紙が現在まで残っている\cite{clara}.
