\documentclass[lualatex,a4paper]{bxjsarticle}
\usepackage{amsmath,amssymb,mathrsfs,braket}
% \usepackage{amsthm,amscd} %定理環境, 可換図式
% \usepackage{ascmac} %screen, itembox環境
\usepackage{graphicx,xcolor}
% \usepackage{fancyhdr,lastpage} %ヘッダー/フッター操作
\usepackage{makeidx} %索引
\usepackage{hyperref} %ハイパーリンク
% \hypersetup{colorlinks=true,linkcolor=blue,citecolor=green}

% 和文フォント
% \usepackage[no-math,ipaex]{luatexja-preset} % IPAexフォント
\usepackage[no-math]{luatexja-fontspec} %Source Han フォント
	\setmainjfont{SourceHanSerifJP}
	\setsansjfont{SourceHanSansJP}
\ltjsetparameter{jacharrange={-2}} % 非ASCII文字がすべて和文と解釈されるのを防ぐ

% 欧文フォント
\setmainfont[Ligatures=TeX]{SourceSerifPro}
% \setmainfont[Ligatures=TeX]{XITS}
\setsansfont[Ligatures=TeX]{SourceSansPro}
% \setmonofont[Ligatures=TeX]{Inconsolatazi4}
\setmonofont[Ligatures=TeX]{SourceCodePro}

% 数式フォント
\usepackage{unicode-math}
\unimathsetup{math-style=ISO,bold-style=ISO}
\setmathfont{LatinModernMath}

% 索引を作成
\makeindex

% jsbook用の設定
\setlength{\textwidth}{\fullwidth} %本文領域を最大化 (余白を小さくする)
\setlength{\evensidemargin}{\oddsidemargin}

% MusiXTeX用設定
\usepackage{musixtex}
\input{musixadd.tex}
\nobarnumbers %小節番号なし
\smallmusicsize %楽譜のサイズを小さく
\newcounter{mycounter} % カウンタの宣言
\setcounter{mycounter}{0} % カウンタの初期化
\newcommand{\useMycounter}[1][]{\refstepcounter{mycounter}{#1}譜例{\themycounter}: }
\newcommand{\musicbegin}{\vspace*{5truemm}\begin{music}}
\newcommand{\musicend}[2]{\end{music}\nopagebreak\begin{center}{\small\useMycounter[\label{#1}]{#2}}\end{center}}

% 小節番号表示用
\newcommand{\shosetu}[1]{
\startbarno=#1
\systemnumbers
\def\writebarno{\llap{\the\barno\barnoadd}}%
\def\raisebarno{2\internote}%
\def\shiftbarno{1.3\Interligne}%
}

% 構造番号
\newcommand{\ind}[2]{$\text{#1}_{\text{#2}}$}

% セクション名を「第n曲」にする
\renewcommand{\thesection}{第\arabic{section}曲}

% 文書データ
\title{ブラームス: 5つの歌曲 作品71}
\author{H.~S.}
\date{2017.05.20-}

\begin{document}

\maketitle

\tableofcontents


\section*{全曲の概要}
\addcontentsline{toc}{section}{全曲の概要}

本歌曲集は1877年3月にWienで作曲された\cite{library}\cite{denki}. 出版は同年7月から8月にかけて, 作品69から作品72までの歌曲シリーズとしてSimrock社から\cite{cd}.
交響曲第1番の初演が1876年11月で, BrahmsがSimrockへその楽譜を送付したのが1877年5月であるから,
交響曲第1番の改訂の最中に作曲されたことになる.
なお, 後述するように第1曲「春は優しい恋の季節」の一節が交響曲第2番 (1877年夏に作曲された) に顔を出す.

前後の歌曲集 (作品70, 作品72) が数年に渡って作曲された歌曲の集まりである (1877年に作曲されたものも多く含まれるが) こと\cite{taiyaku}と比較すると,
この歌曲集はごく短期間にまとまって作られたものであると言える.
内容的にもこの5曲はすべて恋愛事を題材としている.

この歌曲集のなかで最も有名な第5曲「愛の歌」はLudwig Hölty (1748--1776) の詩によっており,
同じ詩に対してSchubertやMendelssohnも付曲している.
Karl Candidus (1817--1872) による第3曲「秘め事」も絶妙な音楽的効果のために評価も高い.
なお, Clara Schumannが本歌曲集に関してBrahmsに感想を述べた手紙が現在まで残っている\cite{clara}.



\section{春は優しい恋の季節 (Es liebt sich so lieblich im Lenze)}

\vspace*{5truemm}
\begin{center}
	{\large Es liebt sich so lieblich im Lenze}\\
	\hspace*{25truemm}Heinrich Heine (1797--1856)
\end{center}
\begin{quote}
\begin{multicols}{2}
	Die Wellen blinken und flie{\ss}en dahin, \\
	es liebt sich so lieblich im Lenze! \\
	Am Flusse sitzet die Sch{\"a}ferin \\
	Und windet die z\"artlichsten Kr\"anze. \\

	Das knospet und quillt und duftet und bl{\"u}ht, \\
	Es liebt sich so lieblich im Lenze! \\
	Die Sch{\"a}ferin seufzt aus tiefer Brust: \\
	"Wem geb' ich meine Kr{\"a}nze?" \\

	Ein Reiter reitet den Flu{\ss} entlang, \\
	er gr\"u{\ss}et so bl{\"u}henden Mutes, \\
	die Sch{\"a}ferin schaut ihm nach so bang, \\
	fern flattert die Feder des Hutes. \\

	Sie weint und wirft in den gleitenden Flu\ss \\
	die sch{\"o}nen Blumenkr{\"a}nze. \\
	Die Nachtigall singt von Lieb' und Ku\ss \\
	es liebt sich so lieblich im Lenze!

	% Die Wellen blinken und flie{\ss}en dahin, \\
	% Welle [女] 波; blinken [動] 光る, 輝く; flie{\ss}en [動] 流れる; dahin [副] そこへ
	波しぶきは輝きながら流れ去っていきます \\
	% Es liebt sich so lieblich im Lenze! \\
	% lieben [動] 愛する; sich [代] 自分自身を (3人称の3格, 4格); lieblich [形] 愛らしい; Lenze [男] 春
	春には愛はひときわ愛らしいのです! \\
	% Am Flusse sitzet die Sch{\"a}ferin \\
	% Fluss [男] 川; sitzen [自] 座る; Sch\"afer [男] 羊飼い (Sch\"aferinは女性形)
	羊飼いの女の子が川縁に座りながら \\
	% Und windet die z\"artlichsten Kr\"anze. \\
	% winden [他] を編む; z{\"a}rtlich [形] 愛情のこもった; Kranz [男] 花輪
	愛情込めて花輪を編んでいます \\

	% Das knospet und quillt und duftet und bl{\"u}ht, \\
	% knospen [自] つぼみをつける; quellen [自] 湧き出る; duften [自] 香る; bl\"uhen [自] 咲いている
	蕾が芽生え, 香りだし, 花が咲く\\
	% Es liebt sichso lieblich im Lenze! \\
	春には愛はひときわ愛らしいのです! \\
	% Die Sch{\"a}ferin seufzt aus tiefer Brust: \\
	% seufzen [自] ため息をつく; aus [前] の中から; tief [形] 深い; Brust [女] 胸
	その女の子が胸の奥からため息をつきます \\
	% "Wem geb' ich meine Kr{\"a}nze?" \\
	% wem [代] 誰に; geben [他] を与える
	「誰に私の花輪を差し上げましょうか?」\\

	% Ein Reiter reitet den Flu\ss entlang, \\
	% Reiter [男] 騎兵; reiten [自] (馬に) 乗る; entlang [後置詞] に沿って
	一人の男性が川沿いに乗馬しながら \\
	% er gr{\"u}{\ss}et so bl{\"u}henden Mutes, \\
	% gr\"u{\ss}en [他] 挨拶する; bl{\"u}hen [自] (花が) 咲いている; Mutes
	女の子の方に溌溂と挨拶をします\\
	% die Sch{\"a}ferin schaut ihm nach so bang, \\
	% shcauen [自] 眺める; bang [形] 気がかりな
	女の子は彼の方を気にして眺めていると \\
	% fern flattert die Feder des Hutes. \\
	% fern [副] 遠い; flattern [自] 飛んでいく, たなびく; Feder [女] 羽; Hut [男] 帽子
	遠くで帽子の羽がはたはたと揺れました \\

	% Sie weint und wirft in den gleitenden Flu\ss \\
	% weinen [自] 泣く; werfen [他] を投げる; gleiten [自] 滑る;
	女の子は泣きながら投げ入れてしまいました \\
	% die sch{\"o}nen Blumenkr{\"a}nze. \\
	% sch\"on [形] 美しい
	流れゆく川面に, その美しい花の環を \\
	% Die Nachtigall singt von Lieb' und Ku\ss \\
	% von [前] から
	小夜啼鳥は愛と接吻について歌います \\
	% es liebt sich so lieblich im Lenze!
	春には愛はひときわ愛らしいのです!
\end{multicols}
\end{quote}

\vspace*{5truemm}

Heinrich Heineの詩による可愛らしい歌曲である.
本作のテクストの出典は「新詩集」 (Neue Gedichte, 1844年) の"Romanzen"節に含まれる23の詩のうち第13番「春」
(Frühling) である\footnote{Charles Stanfordもこの詩に付曲している (作品4-4).}\cite{liedernet}.
この詩自体の作詞は1839年とされている.
なお, BrahmsはHeineの詩にわずか6曲しか付曲していない:
他の歌曲はいずれもこの作品より後期のもので, 作品85-1, 作品85-2, 作品96-1, 作品96-3, 作品96-4である.

有節歌曲形式を下敷きにしているが, 第3句および第4句は自由に変奏されている.
特に第3句では嬰ヘ長調へと移り, 馬の蹄のリズムが三連符で表現される.
それ以外は概してピアノパートは流れるような音型が続き, 留まることのない川の流れや煌きを思わせる.
対する歌唱は民謡風で, シンプルかつ歯切れが良い.
また, カデンツを除いて属和音が回避される傾向にあり, 春の穏やかな陽気を醸し出している.

原調はニ長調, 4分の4拍子であるが, ピアノの序奏が強拍を避けて開始する上に歌唱もアウフタクトで始まるため, やや拍子が取りにくい (譜例\ref{1-1}).

\musicbegin
	\def\nbinstruments{2}%   % パート数
	\setstaffs{1}{2}%        % 下から1番目は2段
	\setclef{1}{6000}%       % 下から1番目はへ音記号
	\generalsignature{+2}%    % 調号は正の値のときシャープの数
	\generalmeter{\meterC}%  % 拍子
	\startextract%
	\nnnotes\cl{*}\enotes
	\setclef{1}{0}\zchangeclefs
	\notesp
		\zmidstaff{\p \ {\small dolce}}%
		\ds\ibu{0}{h}{-2}\islurd{1}{i}\qb{0}{ih}\tbu{0}\qb{0}{f}\ibu{0}{d}{-1}\qb{0}{ded}\tbu{0}\tslur{1}{b}\qb{0}{b}|%
		\ds\ibl{0}{'h}{-2}\isluru{2}{i}\qb{0}{ih}\tbl{0}\qb{0}{f}\ibl{0}{d}{-1}\qb{0}{ded}\tbl{0}\qb{0}{b}&%
		\zcharnote{s}{\hspace*{-8truemm}Anmutig bewegt}%
	\enotes
	\nnnotes\cl{*}\enotes
	\def\atnextbar{\znotes&\centerbar{\cpause}\en}%
	\setclef{1}{6}\zchangeclefs
	\bar
	\notesp
		\ibl{0}{a}{-1}\isluru{1}{a}\qb{0}{aba}\tbl{0}\qb{0}{M}\ibu{0}{K}{-1}\qb{0}{KLK}\tbu{0}\tbsluru{1}{b}\qb{0}{G}|%
		\ibl{0}{'c}{+1}\curve 116\tslur{2}{a}\qb{0}{a}\isluru{2}{i}\qb{0}{ih}\tbl{0}\qb{0}{f}\ibl{0}{d}{-1}\qb{0}{ded}\tbl{0}\qb{0}{b}&%
		\cu{*}\hpause\cu{**}\qp\cu{*}\zcharnote{M}{\scriptsize Die}\qu{d*}
	\enotes
	\bar
	\notesp
		\ibu{0}{K}{0}\ibsluru{1}{K}\qb{0}{KDM}\tbu{0}\qb{0}{K}\ibl{0}{K}{0}\qb{0}{NKb}\tbl{0}\qb{0}{K}|%
		\ibl{0}{'c}{+1}\curve 116\tslur{2}{a}\qb{0}{a}\ibslurd{2}{N}\qb{0}{ih}\tbl{0}\qb{0}{f}\ibl{0}{d}{-1}\qb{0}{ded}\tbl{0}\qb{0}{b}&%
		\zcharnote{N}{\scriptsize wel--}\qu{f*}\zcharnote{N}{\scriptsize len}\qu{h*}
			\zcharnote{N}{\scriptsize blin--}\ql{i*}\ccharnote{N}{\scriptsize ken}\cl{k}\zcharnote{N}{\scriptsize und}\cl{k}
	\enotes
	\bar
	\notesp
		\ibl{0}{L}{-1}\qb{0}{dKN}\tbl{0}\qb{0}{K}\ibl{0}{L}{-1}\qb{0}{dKM}\tbl{0}\curve 111\tbsluru{1}{H}\qb{0}{K}|%
		\ibu{0}{h}{-1}\qb{0}{hji}\tbu{0}\qb{0}{g}\ibu{0}{g}{0}\qb{0}{fih}\tbu{0}\curve 233\tsslur{2}{f}\qb{0}{f}&%
		\zcharnote{N}{\scriptsize flie--}\ql{m*}\ccharnote{N}{\scriptsize {\ss}en}\cl{k}
			\zcharnote{N}{\scriptsize da-}\cl{i}\zcharnote{N}{\scriptsize hin}\qu{h*}\zcharnote{N}{\scriptsize es}\ql{k*}
	\enotes
	\endextract
\musicend{1-1}{作品71-1冒頭}

既に指摘したように, 本作はBrahmsが交響曲第2番で引用している点が特筆される.
その第1楽章第502小節からのフルートの旋律 (譜例\ref{sym2}) は, (基本動機D-Cis-Dから導き出されているとはいえ)
作品71-1の第10小節の動き (譜例\ref{1-10}) とまったく同じである.
Brahmsは自身の手元の交響曲の出版譜にその事実を書き込んでいる.

\musicbegin
\shosetu{10}
\def\nbinstruments{1}%   % パート数
\setstaffs{1}{1}%        % 下から1番目は2段
\setclef{1}{0000}%       % 下から1番目はへ音記号
\generalsignature{+2}%    % 調号は正の値のときシャープの数
% \generalmeter{\meterfrac{2}{4}}%  % 拍子
\startextract%
\Notes\ibu{0}{f}{+3}\islurd{1}{f}\qb{0}{f}\tbu{0}\tslur{1}{h}\qb{0}{h}\sh{g}\cu{g}\cl{i}\enotes
\NOtes\ql{l}\enotes
\Notes\cl{k}\cl{i}\enotes
\bar
\NOtes\hu{h*}\qu{d}\qp\enotes
\endextract
\musicend{1-10}{作品71-1第10小節から}

実際, 本歌曲「春は優しい恋の季節」の描写する空気感は問題の交響曲を満たしているものでもある (やや前者の方が軽いか).
これはBrahmsが交響曲第2番を作曲するにあたって南オーストリアのPörtschachを避暑地に選んでおり,
当地の森や湖の美しさに影響されたからと解釈されている.
BrahmsはBillrothへの手紙の中で「湖上にて」作品59-1の一節を引用してPörtschachを称えている.
他にも, 第1楽章第2主題が有名な「子守歌」作品49-4とよく似た節回しである等,
交響曲第2番とBrahmsの歌曲との関係性は特に顕著なものとなっている.

\musicbegin
	\shosetu{502}
	\def\nbinstruments{1}%   % パート数
	\setstaffs{1}{1}%        % 下から1番目は2段
	\setclef{1}{0000}%       % 下から1番目はへ音記号
	\generalsignature{+2}%    % 調号は正の値のときシャープの数
	\generalmeter{\meterfrac{3}{4}}%  % 拍子
	\startextract%
	\NOtes\ql{p}\sh{o}\ql{o}\ql{p}\enotes
	\bar
	\NOtes\ql{s}\ql{r}\ql{p}\enotes
	\bar
	\NOtes\na{o}\hlp{o*}\enotes
	\bar
	\NOtes\ql{k}\qp\enotes
	\zendextract
\musicend{sym2}{交響曲第2番第1楽章第502小節からのFl}



\section{月に寄せて}



\section{秘め事}



\section{私に行って欲しいの? (Willst du, daß ich geh'?)}

\begin{quote}
	Auf der Heide weht der Wind, \\
	Herzig Kind, herzig Kind, \\
	Willst du, daß trotz Sturm und Graus \\
	In die Nacht ich muß hinaus? \\
	Willst du, daß ich geh'? \\

	Auf der Heid' zu Bergeshöh' \\
	Treibt der Schnee, treibt der Schnee; \\
	Feget Straßen, Schlucht und Teich \\
	Mit den weißen Flügeln gleich. \\
	Willst du, daß ich geh'? \\

	Horch, wie klingt’s herauf vom See \\
	Wild und weh, wild und weh! \\
	An den Weiden sitzt die Fei, \\
	Und mein Weg geht dort vorbei. \\
	Willst du, daß ich geh'? \\

	Wie ist's hier in deinem Arm \\
	Traut und warm, traut und warm; \\
	Ach, wie oft hab' ich gedacht: \\
	So bei dir nur eine Nacht. \\
	Willst du, daß ich geh'?
\end{quote}



\section{愛の歌 (Minnelied)}

\begin{quote}
	Holder klingt der Vogelsang, \\
	Wenn die Engelreine, \\
	Die mein Jünglingsherz bezwang, \\
	Wandelt durch die Haine. \\

	Röter blühen Tal und Au, \\
	Grüner wird der Wasen, \\
	Wo die Finger meiner Frau \\
	Maienblumen lasen. \\

	Ohne sie ist alles tot, \\
	Welk sind Blüt' und Kräuter; \\
	Und kein Frühlingsabendrot \\
	Dünkt mir schön und heiter. \\

	Traute, minnigliche Frau, \\
	Wollest nimmer fliehen; \\
	Daß mein Herz, gleich dieser Au, \\
	Mög' in Wonne blühen!
\end{quote}


\printindex

\end{document}
